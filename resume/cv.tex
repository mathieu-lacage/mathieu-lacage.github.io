\documentclass[a4paper,12pt]{article}

\usepackage{cmbright}
\usepackage[latin1]{inputenc}
\usepackage[T1]{fontenc}
\usepackage[a4paper]{geometry}
\usepackage[dvips]{graphicx}
\usepackage[dvips]{graphics}
\usepackage{times}
\usepackage{latexsym}
\usepackage{url}


\renewcommand{\arraystretch}{1.2}
\pagestyle{empty}


\newcommand{\ligne}[1]{\rule[0.5ex]{\textwidth}{#1}\\}
\newcommand{\activite}[1]{\textit{#1}\ }
  
\newcommand{\interRubrique}{\bigskip}
\newcommand{\styleRub}[1]{\textbf{\large #1}\par}
\newcommand{\indentStd}{\noindent\hspace*{10pt}}



\newenvironment{rubrique}[2][\linewidth]% "rubrique" prend deux arguments
{\styleRub{#2}%       			% le second argument : titre
\ligne{0.5mm}
\setlength{\lenB}{#1}%		       	% Le premier argument :indentation
\setlength{\lenC}{\linewidth}%		% Calculs...
\addtolength{\lenC}{-\lenA}%
\addtolength{\lenC}{-\lenB}%
%\addtolength{\lenC}{-\parindent}%
\addtolength{\lenC}{-19pt}
\indentStd\begin{tabular}[t]{p{\lenB}p{\lenC}}}
{\end{tabular}}

\newenvironment{rubriquesansligne}[2][\linewidth]% "rubrique" prend deux arguments
{\styleRub{#2}%       			% le second argument : titre
\setlength{\lenB}{#1}%		       	% Le premier argument :indentation
\setlength{\lenC}{\linewidth}%		% Calculs...
\addtolength{\lenC}{-\lenA}%
\addtolength{\lenC}{-\lenB}%
\addtolength{\lenC}{-\parindent}%
\addtolength{\lenC}{-19pt}
\indentStd\begin{tabular}[t]{p{\lenB}p{\lenC}}}
{\end{tabular}}

\newlength{\lenA} % indentation au début d'une ligne
\setlength{\lenA}{0.cm}
\newlength{\lenB} % Taille champ dates
\newlength{\lenC} % Taille champ description
%\newlength{\lenD}


\geometry{hmargin={2cm, 2cm}, vmargin={2cm, 2cm}}

\begin{document}


\begin{tabular*}{1\textwidth}{@{\extracolsep{\fill}}lr}
Mathieu Lacage & mathieu.lacage@cutebugs.net\\
\end{tabular*}

\vspace{1cm}
\begin{center}{\huge Software Engineer}\end{center}

\vspace{0.5cm}
\begin{center}
I build secure products backend-side, from
server infrastructure to application-level software
architecture and implementation
\end{center}

\vspace{1cm}
\begin{rubrique}{Technical Experience}

Security
\begin{itemize}
\item Designed a security policy spanning 
    infrastructure and application-level code.

\item Deployed an SSO for identification and authorization across
    all internal backoffice tools, including SSH.

\item Enforced disk-level at-rest encryption of customer data 
    via an internally-managed SSM, HCVault.
\end{itemize}

Infrastructure
\begin{itemize}
\item Scaled from one server to 200+ VMs and servers via Ansible.
\item Zero-downtime scale up of our single database/single table to partionned
    databases per customer, 250 tables, 1TB.
\end{itemize}

Development
\begin{itemize}
\item Web applications with Python for transactional
    use-cases, websocket event streaming, and asynchronous jobs.
    
\item System-level C and C++ software for virtualization
    and simulation.
\end{itemize}

\end{rubrique}

\vspace{1cm}
\begin{rubrique}{Alcmeon, 2011 -- 2023, CTO, co-founder}

As CTO, I focused on:
\begin{itemize}
\item Leading the development of Alcmeon's product across our infrastructure, Machine Learning magic, Python backend, SQL database, and Angular frontend. 

\item Recruiting and managing the engineering team (13) and the enginneering manager in charge of research, development and operations.

\item Defining and implementing a Security Policy.

\end{itemize}

As co-founder, I invested my efforts in product management, customer support, and, pre-sales.
\end{rubrique}

\begin{rubrique}{INRIA, 2005 -- 2011, Software Lead ns-3}
  Built ns-3, an open-source network simulator now used in hundreds of research publications every year. 
 
  \begin{itemize}
  \item Designed and implemented core APIs: object model, network packets, event scheduler
  \item Implemented models for UDP/IP/ICMP, MAC/PHY Wi-Fi network protocols
  \item Integrated unmodified real-world network protocol implementations 
    via a virtualization framework optimized for simulation environments.
  \end{itemize}

  But also:
  \begin{itemize}
  \item Recruited, and managed local development team (5).
  \item Relocated to University of Washington for 10 months to initiate collaboration with US team-mates.
  \item Evangelized use of ns-3 within other research institutes through
    presentations and seminars.
  \item Published as main author (4) and co-author (3) papers on the design of ns-3 and some of its models.
  \end{itemize}

\end{rubrique}

\vspace{0.5cm}
\begin{rubrique}{INRIA, 2003 -- 2005, Software Engineer}

  Designed and built software for network research teams:  
  \begin{itemize}
  \item Yans, a C++ event-driven simulator,
  \item NEPI, a python tool used to describe, deploy, and control networking
    experiments on hundreds of hosts distributed all over the planet.
  \end{itemize}

  Provided software mentoring to research projects involved in 
  bio-reactor chemical reaction control and medical image analysis.

\end{rubrique}

\vspace{0.5cm}
\begin{rubrique}{Sigma-Designs, 2001 -- 2003, Software Engineer}

  Implemented DVD navigation control software for the video
  decompression chips that were developped by Sigma-Designs and
  sold to OEMs to build consumer DVD players.

\end{rubrique}

\vspace{0.5cm}
\begin{rubrique} [4cm] {Education}
  2006 -- 2010 & Ph.D. at University of Nice,
  \emph{Experimentation Tools for Networking Research}, under supervision from Walid Dabbous\\
  1998 -- 2001 & Engineer at Telecom ParisTech (ENST), Software Engineering, Networking, Micro-Electronics \\
\end{rubrique}

\end{document}

